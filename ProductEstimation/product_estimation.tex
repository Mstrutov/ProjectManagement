\documentclass[a4paper,8pt]{article}

% Encoding.
\usepackage{geometry}
\usepackage[T2A]{fontenc}
\usepackage[utf8]{inputenc}
\usepackage[english,russian]{babel}

% Code insertion.
\usepackage[outputdir=build]{minted}

% Math functions.
\usepackage{amsmath}

% Image insertion.
\usepackage{svg}

% No line breaks.
\usepackage[none]{hyphenat}

\title{Задание 5: экспертная и UCP-оценка проекта ``e-гриб''}
\author{
    \begin{tabular}[t]{c@{\extracolsep{8em}}c} 
        Афанасов Артём     & Смирнов Александр \\
        &\\ 
        Струтовский Максим & Феодор Жилкин
    \end{tabular}
}

\date{\today}

\begin{document}

\maketitle

\section*{UCP-оценка проекта:}

\subsection*{UAW}
\begin{itemize}
  \item Server -- 2
  \item Client -- 2
  \item API -- 1
  \item GUI -- 3
  \item DBs -- 2
  \item $UAV =  2 + 2 + 1 + 3 + 2 = 10$
\end{itemize}

\subsection*{UUCW}
\begin{itemize}
    \item Покупка ЧГ \footnote{Чайный гриб} 4-7
    \item Логин/Регистрация в приложении -- 1-3
    \item Обращение в поддержку -- 1-3
    \item Получение информации о ЧГ --  1-3
    \item $UUCW = 3 \times 5 + 10 = 25$
\end{itemize}

\subsection*{TCF}
\begin{itemize}
    \item Распределенная система -- 1
\item Время отклика/цели производительности  -- 5
\item Эффективность конечного пользователя -- 3
\item Внутренняя сложность обработки -- 2
\item Переиспользование кода -- 1 
\item Простота установки -- 5
\item Простота использования -- 5
\item Переносимость на другие платформы -- 3
\item Техническое обслуживание -- 5
\item Параллельная обработка  -- 1
\item Безопасность -- 2
\item Доступ к сторонним приложениям  -- 5
\item Обучение пользователей -- 3
\item $SUM = 1\times2.0 + 5\times1.0 + 3\times1.0 + 2\times1.0 + 1\times1.0 + 5\times0.5 + 5\times0.5 + 3\times2.0 + 5\times1.0 + 1\times1.0 + 2\times1.0 + 5\times1.0 + 3\times1.0 = 37$
\item $TCF =  0.6 + \frac{37}{100} = 0.97$
\end{itemize}

\subsection*{ECF}
    \begin{itemize}
    \item Знакомство с используемым процессом разработки -- 2
    \item Опыт использования приложения -- 1
    \item Объектно-ориентированный опыт команды -- 5
    \item Возможности ведущего аналитика -- 4
    \item Мотивация команды -- 3
    \item Стабильность требований -- 3
    \item Частичная занятость -- 2
    \item Сложность ЯП -- 1
    \item $SUM = 2 \times 1.5 + 1\times0.5 + 5\times1.0 + 4\times0.5 + 3\times1.0 + 3\times2.0 + 2\times(-1) + 1\times(-1) = 16.5$
    \item $ECF = 1.4 + (-0.03 \times 16.5) = 0.905$
\end{itemize}

\subsection*{Итого}


    \begin{itemize}
        \item $UCP = (UAW + UUCW) \times TCF \times ECF = (10 + 25) \times 0.97 \times (0.905) = 30.72$
        \item Затраты = $UCP \times 30 = 921.6$ человеко-часов
        \item Штат сотрудников (программистов) -- 5
        \item $\frac{921.6}{5 \times 8} = 23$ дня
    \end{itemize}

\section*{Экспертная оценка}

    Исходя из полученных результатов UCP-оценки проекта, можно сделать вывод, что бюджет и сроки проекта поставлены корректно: 5 программистов будут разрабатывать ПО проекта в течение 23 дней, что удовлетворяет бюджету и срокам.

    На разработку ПО выделено 3 млн. рублей, что является правильным решением, поскольку большая часть бюджета должна быть потрачена на выращивание ЧГ, поддержку, рекламную компанию. Важным моментом является то, что UCP оценка произведенная даже до начала выполнения проекта дает весьма точную оценку затрат, вне зависимости от уровня навыков команды разработчиков. Также, так как наш программный комплекс не является очень сложным и программисты нам нужны всего-лишь на один месяц нам надо изначально точно рассчитать требуемое количество, во избежание последующего до найма (как известно по книге Брукса ``Мифический человеко-месяц'' добавление программистов в готовую команду -- путь в никуда). 

    UCP оценка была произведена опытными людьми с учетом всех технических факторов и факторов окружающей среды, что говорит о ее высокой точности. Тем не менее, на ранних этапах жизни проекта допускается ее небольшое изменение.


\end{document}

