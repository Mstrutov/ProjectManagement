
\documentclass[a4paper,8pt]{article}

% Encoding.
\usepackage{geometry}
\usepackage[T2A]{fontenc}
\usepackage[utf8]{inputenc}
\usepackage[english,russian]{babel}

% Code insertion.
\usepackage[outputdir=build]{minted}

% Math functions.
\usepackage{amsmath}

% Image insertion.
\usepackage{svg}

% No line breaks.
\usepackage[none]{hyphenat}

\title{Задание 6: Технико-коммерческое предложение ``e-гриб''}
\author{
    \begin{tabular}[t]{c@{\extracolsep{8em}}c}
        Афанасов Артём     & Смирнов Александр \\
        &\\
        Струтовский Максим & Феодор Жилкин
    \end{tabular}
}

\date{\today}
\begin{document}

\maketitle

\section{Информация о компании}
Содержит информацию для заказчика, чтобы он понимал с кем работать, особенно в долгосрочных отношениях:
\begin{itemize}
  \item структура собственности
  \item финансово-экономические показатели для определения стабильности компании, например, оборот и прибыль. Информацию публикуют на сайте или раскрывают конкретно заказчику.
  \item система менеджмента качества компании (например, подтверждение того, что компания имеет систему ИСО 9001 (в инете ИСО 9000 :/) например, система CMMI (если правильно услышал), либо какая-то внутренняя система)
  \item процедуры управления проектами -- описать, как мы умеем разбираться с проектами. Тайм и материал или фикс прайс влияет насколько заказчик погружен на процесс управления => зависит полнота описания этого пункта. Выбор Agile (scrum or Kannan), простой  водопад и т.п.
  \item квалификация и численность персонала (Сколько (+ их квалификация) разработчиков на джава, тестеров, руководителей проекта. Конкретные цифры или процент от компании)
  \item квалификация по рынкам (Указать опыт работы с конкретными областями, например, банки; Описывается общая квалификация компании в целом по всем областям, с которыми компания работала (Про области, соответствующие заказчику, давать информацию подробнее))
  \item заказчики -- общая инфа, чтобы заказчик мог проверить, что мы делаем работу (Крупные, мелкие, какие регионы, какие предметные области, контакты конкретных лиц и компаний предыдущих заказчиков, с которыми настоящий заказчик может связаться сам)
\end{itemize}

\section{Рамки проекта}
\begin{itemize}
  \item рамки проекта (обычно описывается заказчиком)
  \item бизнес цели -- для чего вообще проект
  \item описание продукта -- кратко
  \item список продуктов, компонент и т.д.
  \item цели проекта:
    \begin{itemize}
      \item Приложения под платформы:
          \begin{itemize}
              \item iOS;
              \item Android;
              \item Web;
              \item 3500 заказов ЧГ в день;
          \end{itemize}
      \item Произодство ЧГ:
          \begin{itemize}
              \item Ферма по производству ЧГ;
              \item 4000 ЧГ в день;
          \end{itemize}
      \item Организация доставки ЧГ до клиента:
          \begin{itemize}
              \item 3 склада в разных концах Санкт-Петербурга;
              \item Аутсорс доставки из складов;
          \end{itemize}
      \item Чистая прибыль 10 млн. руб. в месяц.
    \end{itemize}
  \item требования (не всегда проработаны на этапе подготовки предложения)
\end{itemize}

\section{Описание решения проекта}

\begin{itemize}
  \item архитектура (даже для не )
  \item основные вехи и результаты в более глобальном взгляде, чем раньше рассматривались в “основных фазах”
  \item Мокапы основных экранов (если применимо)/спецификация API
  \item спецификация среды разработки: описывать её, если передаем код проекта для дальнейшего его. н-р, Jave, многопроцессорный сервер, ...
  \item требования к среде выполнения: какие нужны аппаратные ресурсы, программные средства для работы проекта у клиентов
\end{itemize}

\section{Процедуры управления проектом}

\begin{itemize}
  \item Планируемые отчеты и демонстрации
  \item организация взаимодействия/коммуникации
  \item описание доставки и приемки
  \item сжатый план тестирования (опц.)
  \item подход к подбору персонала (опц.) -- заказчику важно знать, как мы с этим справимся, т.к. тяжело найти качественные кадры
\end{itemize}

\section{Команда}
\begin{itemize}
  \item ~ состав
  \item квалификация (относительно контекста)
  \item сертификаты (иногда обязательно: н-р, в гос заказах)
\end{itemize}

\section{Структурная декомпозиция работ - скопипастить}
копипаста

\section{Оценка проекта - скопипастить}
копипаста

\section{Условия оплаты}
надо что-то придумать

% структурную декомпозицию работ -- WBS (Word Breakdown Structure).
% функции:
% не специфицирует порядок выполнения
% делит проект
% описывает рамки
% помогает с оценкой проекта, графиком работ
% как создать:
% нужно создать дерево. Листья -- работа, вершины -- группировка
% лист описывает работу так, что её легко оценить и спланировать
% должен описывать работу ясно и точно
% соответствующая работа должна быть легко планируема (можно юзать для оценки)
% у работы известны ответственные лица
% в начале можно разбить на главные элементы (как проект организован) или по этапам
% для каждого листа в промежуточном дереве смотрим: выполняются критерии листа? Если нет, декомпозируем.
% оценку проекта
% условия оплаты


\end{document}
